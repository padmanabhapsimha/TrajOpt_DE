% ************************** Thesis Abstract *****************************
% Use `abstract' as an option in the document class to print only the titlepage and the abstract.
\begin{abstract}
The problem of optimal transfer between planetary parking orbits using low thrust electric propulsion systems has been solved in this study. In general, low thrust interplanetary transfers ignore the gravity fields of the target bodies (example - Earth and Mars). This results in only optimal heliocentric transfers between orbits of the planets. In this study, first, the following transfers have been optimized (i) Transfers between heliocentric orbits of planets. (ii) Transfers from geosynchronous transfer orbit to geosynchronous orbits by ignoring the gravity fields of other bodies. Then, a unified approach that includes the gravity fields of the target bodies as well as the Sun has been developed. This approach has been applied to optimize, (iii) Transfers between parking orbits around Earth and Moon. (iv) Transfers from Earth parking orbits to Mars parking orbits.\\
Low thrust transfers require the solution to continuous optimal control problems. The indirect approach to optimal control using the Pontryagin's minimum principle has been followed. This results in the formulation of a two point boundary value problem. The control variables are expressed as functions of the costates. Here, the initial values of the costates are the unknown variables. They are determined using differential evolution, a search based global optimization technique.\\
Many studies available in literature have reported energy optimal trajectories. The current fuel optimal formulation produces better solutions. This is due to the capability of differential evolution to automatically introduce appropriate coasting periods which in turn reduces the expenditure of propellant. For fuel optimal transfers between low altitude planetary parking orbits, the current formulation avoids the need for multi level optimization.\\ 
Nuclear and solar electric power models and their impact on spacecraft trajectories have been investigated. It has been observed that nuclear electric propulsion offers significant advantages when compared to solar electric propulsion for outward travel in the solar system.\\
A parallelized code has been developed in standard C++17. The efficiency of the parallelization has been tested. The differential evolution algorithm has been benchmarked against a wide variety of test problems.\\

\end{abstract}
