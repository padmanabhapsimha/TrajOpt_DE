%*******************************************************************************
%****************************** Second Chapter *********************************
%*******************************************************************************

\chapter{Literature Survey and Motivation}

\ifpdf
    \graphicspath{{Chapter2/Figs/Raster/}{Chapter2/Figs/PDF/}{Chapter2/Figs/}}
\else
    \graphicspath{{Chapter2/Figs/Vector/}{Chapter2/Figs/}}
\fi
In this chapter, the work regarding low thrust spacecrafts and related trajectory optimization is briefly presented. First, some papers on spacecraft equipped with low thrust systems are reviewed followed by papers on trajectory design and optimization for low thrust spacecraft. Papers on optimal control theory are also reviewed along with new developments in the field of spacecraft trajectory optimization. Papers by \cite{nah_fuel-optimal_2001} and \cite{genta_optimal_2016} are of special interest for complete transfers between planetary parking orbits. Rather than focus on the solution strategy adopted by the respective authors, these two papers are taken as benchmarks to compare the current results against. Both the papers attempt the solution to the problem of transferring a spacecraft from a low Earth orbit to a low Mars orbit. The implications of their problem formulation and solution strategy will be discussed in detail.

\section{Low thrust spacecraft}
Some of the earliest investigations on electrically propelled spacecraft were conducted by Robert Goddard and Konstantin Tsiolkovsky. \cite{stuhlinger_ion_1964} presented the first systematic analysis of electric propulsion systems. Modern electric propulsion systems have significantly evolved since the 1950s. The original designs used low boiling metals like Cesium and Mercury. Recent ion and hall effect thrusters utilize Xenon as the propellant due to it's large atomic size and ease of ionization. Since the 1960s, electric propulsion programs were established in NASA Glenn Research Center and the Jet Propulsion Laboratory. There were also several institutes in Russia established to develop electric thrusters. The Russians were the first to utilize electric propulsion for their communication satellites for station keeping. The fall of the Soviet Union prompted the introduction of their stationary plasma thrusters (SPT) series of electric propulsion systems for use on Western spacecrafts. The NTRS-NASA report by \cite{sankovic_performance_1994} presents the results of extensive tests performed on the SPT-100 thruster conducted at NASA's Lewis Research Center. Several variants of this thruster have flown on ESA missions including SMART-1 which performed a complete lunar transfer from Earth orbit \citep{kugelberg_accommodating_2004}. Commercial communication spacecraft as of today have routinely used electric propulsion for both station keeping as well as for orbit raising to geosynchronous orbit. ESA's GOCE spacecraft was able to remain in a low Earth orbit at $235km$ for over eleven months due to the on board electric thruster continuously combating atmospheric drag.\\
NASA's Deep Space 1 was a technology demonstration probe that performed flybys past an asteroid and a comet. This was the first probe too utilize solar electric propulsion as the primary source of propulsion. The mission design details have been presented by \cite{rayman_mission_1999}. NASA's Dawn spacecraft was the successor to the Deep Space 1 probe and successfully demonstrated a velocity impulse of over $10kms^{-1}$ in flight \citep{rayman_dawn_2006}. JAXA spacecraft like the Hayabusa have also demonstrated the use of electric propulsion for deep space missions. \cite{williams_benefits_1997} and \cite{circi_mars_2004} discuss various solar power models for the purpose of solar electric propulsion along with some mission design details. \cite{hoskins_30_2013} discuss the development of electric thrusters in Aerojet Rocketdyne as well as from other organizations.
\section{Trajectory design and optimization with low thrust spacecraft}
\cite{edelbaum_propulsion_1961}, in his seminal paper presented closed form analytic solutions for maneuvers including those by low thrust systems to perform large changes in orbital elements. Some of his results included closed form expressions for transfers between inclined circular orbits of different radii. \cite{arthur._e_applied_1975} describes a heliocentric Earth-Mars transfer using direct optimal control. This is one of the first sources that presents the use of direct optimal control theory for electric propulsion spacecraft. \cite{rauwolf_near-optimal_1996} applies the genetic algorithm to obtain a near optimal trajectory for the same problem. \cite{kim_continuous_2005}, in his PhD dissertation solves this problem by using the indirect approach to optimal control theory by using an adaptive algorithm combining search based and gradient methods along with homotopy and symmetry methods to obtain the solution to the resulting two point boundary value problem. The results compare well with the previously obtained results from literature.\\
\cite{racca_capability_2001} highlights the advantages of electric propulsion systems over lower specific impulse high thrust chemical thrusters with regard to missions including gravity assists. Narrowing of launch windows due to relative geometry required for planetary positions restricts the usefulness of the high thrust missions. Additionally, long cruise times lead to greatly increased operating costs and put extra loads on the spacecraft systems. Solar electric propulsion is explored as a viable alternative for such type of interplanetary missions. \cite{yam_preliminary_2004} presents design strategies for low thrust missions with gravity assists using shape based representations for the spacecraft trajectories. This is a low fidelity approach and it has the assumption of the knowledge of the structure of the control. Even with the shortcomings of this technique, it is valuable to obtain an initial guess for the mission profile before a detailed trajectory optimization run. \cite{patel_maximizing_2006} analyses the impact of taking different cost functionals for the trajectory optimization. External variables from the structural aspects of the spacecraft and the launch vehicle performance parameters are integrated into the formulation. Trade studies have been performed to analyze the effect of variations in the excess velocity provided, propulsion system power availability and specific impulse.\\
\cite{quarta_time-optimal_2013} utilizes a hybrid scheme involving genetic algorithm and gradient descent methods to obtain time optimal solutions to solar electric propulsion missions in the heliocentric frame of reference. The obtained solutions for solar electric propulsion spacecraft are compared with solar sail missions. It is reported that the optimal time is to be taken as the lower bound for a feasible solution in the fuel optimal framework.
\cite{betts_survey_1998} presents a survey of different techniques for the solution to trajectory optimization problems. Issues regarding the difficulty of obtaining a solution to the two point boundary value problem arising from indirect methods to optimal control are highlighted. \cite{chiravalle_nuclear_2008} compares nuclear thermal and nuclear electric propulsion systems for interplanetary transfers. Direct methods are used where the control is parametrized as a polynomial with direct collocation to obtain the coefficients. Some disadvantages with this approach would lie in the generation of orbits with multiple revolutions with spacecrafts unless suitable coordinate systems are utilized for each individual problem.\\
\cite{casalino_indirect_2012} uses indirect optimal control applied to electric propulsion for asteroid deflection missions. In this paper, the authors specified the thrust and coast arcs a priori with successive additions of coast arcs to move the solution to the optimum. A patched conic model is used with the assumption that the spacecraft leaves the Earth by means of chemical propulsion systems.
\cite{kluever_optimal_1997} applies indirect optimal control for trajectory optimization using a fixed thrust-coast-thrust sequence. Analytic expressions are used to replace the initial and final circular orbits around the moon as they would slow down the numerical solution procedure and also lead to stalling in convergence for most numerical methods. The circular restricted three body formulation is used in this paper.\\
\cite{nah_fuel-optimal_2001} solves the problem of Earth-Mars transfer using a variable specific impulse nuclear electric propulsion system. The transfer is performed from a low Earth orbit to a low Mars orbit. Coordinate frame transformations are performed to reduce numerical sensitivity during geocentric and areocentric phases. The problem is solved using indirect optimal control and a gradient descent method. Costate jump conditions have been derived at the frame switching locations. This approach is suitable for realistic mission design and optimization as all the relevant parameters can be included in the mathematical model along with various stages of the trajectory. The authors note that at the middle of the interplanetary phase, the specific impulse rises sharply and as the system is power limited, there is a large drop in the magnitude of the thrust.\\
\cite{genta_optimal_2016} solve a similar problem as \cite{nah_fuel-optimal_2001}. The difference is that the initial and final orbits about Earth and Mars are much lower. The trajectory optimization is split into multiple phases which are then patched together. The splitting is accommodated as a parameter and a parameter optimization is performed to get a globally optimal trajectory. This paper also solves the problem using solar electric propulsion. Earth-Saturn and Earth-Mercury transfers are also evaluated. The main drawback of these two papers is that the thruster is assumed to be capable of a huge thrust and specific impulse range. It is speculated that this is done to avoid solving a fuel optimal problem which would lead to a thrust-coast switching control structure. This would lead to huge difficulties in obtaining converged solutions to the boundary value problem. It is mentioned that during the phases of unrealistically high specific impulse at low thrust levels, the thrusters can be switched off leading to a coast that can potentially save fuel. This strategy is certainly viable but the extent to which the obtained solution will have to be modified to meet the final objectives is still an unknown. In the current work, a similar problem is attempted using a full ephemeris model. A single operating point is selected from the thrust and specific impulse range of the thruster used by \cite{genta_optimal_2016}. In the current work, the problem has been solved by a fuel optimal framework with only one split of the trajectory done at the end of Earth escape. It has been consistently observed that the fuel optimal solution with a fixed thruster operating point can yield better solutions than the variable specific impulse problem solved by \cite{genta_optimal_2016}.
\section{Recent developments}
\cite{anderson_role_2009} describe trajectories in the three body framework. The impact of the invariant manifolds on the dynamics of the spacecraft is investigated. It is shown that even with no prior information, trajectory optimization algorithms tend to move towards trajectories that utilize the invariant manifolds to traverse resonances. \cite{sanchez-sanchez_real-time_2018} trains deep neural networks to perform real time trajectory optimization for landing maneuvers assuming perfect information on the lander's state is known. This has the advantage of not having to solve the trajectory optimization problem by direct or indirect means thus reducing the computational requirements that is needed on board real systems. It is reported that large sets of initial states are possible to be dealt with using this method producing near optimal results. \cite{perez-palau_fuel_2018} solve the low thrust fuel optimization problem for Earth-Moon transfer using indirect optimal control. The bi-circular restricted four body problem framework is utilized. Homotopy methods like numerical continuation have been demonstrated to be unsuitable for this purpose due to the numerical sensitivity involved. Massive exploration of the initial costate space is performed and the resulting trajectories are classified into different families based on their shape, transfer duration and fuel fraction. This search is done by backward numerical integration in order to exactly satisfy the final orbit conditions. Further analysis with dynamical systems theory is also presented. This work can yield valuable insights on initial guesses and types of trajectories to search for in a real mission design scenario. The authors are yet to transpose these results to a full ephemeris model.

\section{Motivation for the current study}
There are several drawbacks with the approaches used in literature. Direct optimal control leads to very large nonlinear programming problems. For transfers between planetary parking orbits, the NLP can comprise of several thousand variables. Indirect approaches using gradient descent methods have the disadvantage of having very small radius of convergence. This leads to the requirement of very good initial guesses. The current work applies differential evolution, a search based global optimization method with a large radius of convergence to eliminate the need for a good initial guess on the unknown costate variables.\\
One more important drawback of approaches used in literature on transfers between planetary parking orbits is that typically energy optimal control is performed in order to reduce the numerical sensitivity of the boundary value problem. This leads to suboptimal results as the fuel optimal solution typically does not coincide with the energy optimal result. Several approaches also utilize thrusters with variable thrust and variable specific impulse along with the energy optimal formulation that leads to values that are unrealistic. This also leads to the requirement of a large operating range for the thruster which may not be physically realizable. The current fuel optimal formulation can generate better results with fixed operating point electric propulsion systems and is thus better representative of realistic thruster designs. Even the variable specific impulse, fuel optimal formulation (presented in Appendix \ref{app3}) requires only a few fixed operating points in comparison to the continuous range that is required of the results in literature.\\
Transfers between planetary parking orbits are achieved by splitting the trajectory into three phases and using a higher level optimization routine to optimize the splitting locations. Here, the trajectory is split only into two phases and a strategy is developed to eliminate the need for any sort of multi-level optimization. This helps in reducing computational time.

% Uncomment this line, when you have siunitx package loaded.
%The SI Units for dynamic viscosity is \si{\newton\second\per\metre\squared}.


\nomenclature[z-NEP]{NEP}{Nuclear Electric Propulsion}
\nomenclature[z-SEP]{SEP}{Solar Electric Propulsion}
\nomenclature[z-BVP]{BVP}{Boundary Value Problem}
\nomenclature[z-TPBVP]{TPBVP}{Two Point Boundary Value Problem}
\nomenclature[z-GA]{GA}{Genetic Algorithm}
\nomenclature[z-CR3BP]{CR3BP}{Circular Restricted Three Body Problem}
\nomenclature[z-BR4BP]{BR4BP}{Bi-circular Restricted Four Body Problem}